%!TEX program = xelatex
% Encoding: UTF8
% SEIKA 2015


% Chapter 2 TutorialsHow to ...
% Section 2.1

% \section{综述}

\textbf{综述}

\textbf{MNIST For ML Beginners}

If you're new to machine learning, we recommend starting here. You'll learn about a classic problem, handwritten digit classification (MNIST), and get a gentle introduction to multiclass classification.

如果你是机器学习领域的新手, 我们推荐你从本文开始阅读. 本文通过讲述一个经典的问题, 手写数字识别 (MNIST), 让你对多类分类 (multiclass classification) 问题有直观的了解.

\hyperref[MINIST_beginner]{阅读该教程 | View Tutorial}

\textbf{Deep MNIST for Experts}

If you're already familiar with other deep learning software packages, and are already familiar with MNIST, this tutorial with give you a very brief primer on TensorFlow.

如果你已经对其它深度学习软件比较熟悉, 并且也对 MNIST 很熟悉, 这篇教程能够引导你对 TensorFlow 有初步了解.



\hyperref[MINIST_pros]{阅读该教程 | View Tutorial}

\textbf{TensorFlow Mechanics 101}

This is a technical tutorial, where we walk you through the details of using TensorFlow infrastructure to train models at scale. We use again MNIST as the example.

这是一篇技术教程, 详细介绍了如何使用 TensorFlow 架构训练大规模模型. 本文继续使用MNIST 作为例子.

View Tutorial

\textbf{Convolutional Neural Networks}

An introduction to convolutional neural networks using the CIFAR-10 data set. Convolutional neural nets are particularly tailored to images, since they exploit translation invariance to yield more compact and effective representations of visual content.

这篇文章介绍了如何使用 TensorFlow 在 CIFAR-10 数据集上训练卷积神经网络. 卷积神经网络是为图像识别量身定做的一个模型. 相比其它模型, 该模型利用了平移不变性(translation invariance), 从而能够更更简洁有效地表示视觉内容.

View Tutorial

\textbf{Vector Representations of Words}

This tutorial motivates why it is useful to learn to represent words as vectors (called word embeddings). It introduces the word2vec model as an efficient method for learning embeddings. It also covers the high-level details behind noise-contrastive training methods (the biggest recent advance in training embeddings).

本文让你了解为什么学会使用向量来表示单词, 即单词嵌套 (word embedding), 是一件很有用的事情. 文章中介绍的 word2vec 模型, 是一种高效学习嵌套的方法. 本文还涉及了对比噪声(noise-contrastive) 训练方法的一些高级细节, 该训练方法是训练嵌套领域最近最大的进展.

View Tutorial

\textbf{Recurrent Neural Networks}

An introduction to RNNs, wherein we train an LSTM network to predict the next word in an English sentence. (A task sometimes called language modeling.)

一篇 RNN 的介绍文章, 文章中训练了一个 LSTM 网络来预测一个英文句子的下一个单词(该任务有时候被称作语言建模).

View Tutorial

\textbf{Sequence-to-Sequence Models}

A follow on to the RNN tutorial, where we assemble a sequence-to-sequence model for machine translation. You will learn to build your own English-to-French translator, entirely machine learned, end-to-end.

RNN 教程的后续, 该教程采用序列到序列模型进行机器翻译. 你将学会构建一个完全基于机器学习,端到端的\emph{英语-法语}翻译器.

View Tutorial

\textbf{Mandelbrot Set}

TensorFlow can be used for computation that has nothing to do with machine learning. Here's a naive implementation of Mandelbrot set visualization.

TensorFlow 可以用于与机器学习完全无关的其它计算领域. 这里实现了一个原生的 Mandelbrot 集合的可视化程序.

View Tutorial

\textbf{Partial Differential Equations}

As another example of non-machine learning computation, we offer an example of a naive PDE simulation of raindrops landing on a pond.

这是另外一个非机器学习计算的例子, 我们利用一个原生实现的偏微分方程, 对雨滴落在池塘上的过程进行仿真.

View Tutorial

\textbf{MNIST Data Download}

Details about downloading the MNIST handwritten digits data set. Exciting stuff.

一篇关于下载 MNIST 手写识别数据集的详细教程.

View Tutorial

\textbf{Image Recognition}

How to run object recognition using a convolutional neural network trained on ImageNet Challenge data and label set.

View Tutorial

We will soon be releasing code for training a state-of-the-art Inception model.

我们将毫无保留地发布已经选训练好的, 目前最先进的 Inception 物体识别模型.

Deep Dream Visual Hallucinations

Building on the Inception recognition model, we will release a TensorFlow version of the Deep Dream neural network visual hallucination software.

COMING SOON