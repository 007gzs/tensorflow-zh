\section{Introduction}

\begin{quotation}
Let's get you up and running with TensorFlow!

But before we even get started, let's peek at what TensorFlow code looks like in the Python API, so you have a sense of where we're headed.

Here's a little Python program that makes up some data in two dimensions, and then fits a line to it.
\end{quotation}

\begin{lstlisting}
import tensorflow as tf
import numpy as np

# Create 100 phony x, y data points in NumPy, y = x * 0.1 + 0.3
x_data = np.random.rand(100).astype("float32")
y_data = x_data * 0.1 + 0.3

# Try to find values for W and b that compute y_data = W * x_data + b
# (We know that W should be 0.1 and b 0.3, but Tensorflow will
# figure that out for us.)
W = tf.Variable(tf.random_uniform([1], -1.0, 1.0))
b = tf.Variable(tf.zeros([1]))
y = W * x_data + b

# Minimize the mean squared errors.
loss = tf.reduce_mean(tf.square(y - y_data))
optimizer = tf.train.GradientDescentOptimizer(0.5)
train = optimizer.minimize(loss)

# Before starting, initialize the variables.  We will 'run' this first.
init = tf.initialize_all_variables()

# Launch the graph.
sess = tf.Session()
sess.run(init)

# Fit the line.
for step in xrange(201):
    sess.run(train)
    if step % 20 == 0:
        print(step, sess.run(W), sess.run(b))

# Learns best fit is W: [0.1], b: [0.3]
\end{lstlisting}

The first part of this code builds the data flow graph. TensorFlow does not actually run any computation until the session is created and the run function is called.

To whet your appetite further, we suggest you check out what a classical machine learning problem looks like in TensorFlow. In the land of neural networks the most "classic" classical problem is the MNIST handwritten digit classification. We offer two introductions here, one for machine learning newbies, and one for pros. If you've already trained dozens of MNIST models in other software packages, please take the red pill. If you've never even heard of MNIST, definitely take the blue pill. If you're somewhere in between, we suggest skimming blue, then red.

% Add pics and links here

If you're already sure you want to learn and install TensorFlow you can skip these and charge ahead. Don't worry, you'll still get to see MNIST -- we'll also use MNIST as an example in our technical tutorial where we elaborate on TensorFlow features.

Recommended Next Steps

Download and Setup
Basic Usage
TensorFlow Mechanics 101

本章的目的是让你了解和运行 TensorFlow!

在开始之前, 让我们先看一段使用 Python API 撰写的 TensorFlow 示例代码,
让你对将要学习的内容有初步的印象.

这段很短的 Python 程序生成了一些三维数据, 然后用一个平面拟合它.

\begin{lstlisting}
import tensorflow as tf
import numpy as np

# 使用 NumPy 生成假数据(phony data), 总共 100 个点.
x_data = np.float32(np.random.rand(2, 100)) # 随机输入
y_data = np.dot([0.100, 0.200], x_data) + 0.300

# 构造一个线性模型
#
b = tf.Variable(tf.zeros([1]))
W = tf.Variable(tf.random_uniform([1, 2], -1.0, 1.0))
y = tf.matmul(W, x_data) + b

# 最小化方差
loss = tf.reduce_mean(tf.square(y - y_data))
optimizer = tf.train.GradientDescentOptimizer(0.5)
train = optimizer.minimize(loss)

# 初始化变量
init = tf.initialize_all_variables()

# 启动图 (graph)
sess = tf.Session()
sess.run(init)

# 拟合平面
for step in xrange(0, 201):
    sess.run(train)
    if step % 20 == 0:
        print step, sess.run(W), sess.run(b)

# 得到最佳拟合结果 W: [[0.100  0.200]], b: [0.300]
\end{lstlisting}

为了进一步激发你的学习欲望, 我们想让你先看一下 TensorFlow 是如何解决一个经典的机器学习问题的. 在神经网络领域, 最为经典的问题莫过于 MNIST 手写数字分类问题. 我们准备了两篇不同的教程, 分别面向机器学习领域的初学者和专家. 如果你已经使用其它软件训练过许多MNIST 模型, 请阅读高级教程 (红色药丸链接). 如果你以前从未听说过 MNIST, 请阅读初级教程(蓝色药丸链接). 如果你的水平介于这两类人之间, 我们建议你先快速浏览初级教程, 然后再阅读高级教程.

% <div style="width:100%; margin:auto; margin-bottom:10px; margin-top:20px; display: flex; flex-direction: row">
%  <a href="tensorflow-zh/SOURCE/tutorials/mnist_beginners.md" title="面向机器学习初学者的 MNIST 初级教程">
%    <img style="flex-grow:1; flex-shrink:1; border: 1px solid black;" src="../images/blue_pill.png" alt="面向机器学习初学者的 MNIST 初级教程" />
%  </a>
%  <a href="tensorflow-zh/SOURCE/tutorials/mnist_pros.md" title="面向机器学习专家的 MNIST 高级教程">
%    <img style="flex-grow:1; flex-shrink:1; border: 1px solid black;" src="../images/red_pill.png" alt="面向机器学习专家的 MNIST 高级教程" />
%  </a>
% </div>
% <p style="font-size:10px;">图片由 CC BY-SA 4.0 授权; 原作者 W. Carter</p>

如果你已经下定决心, 准备学习和安装 TensorFlow, 你可以略过这些文字, 直接阅读
后面的章节. 不用担心, 你仍然会看到 MNIST -- 在阐述 TensorFlow 的特性时,
我们还会使用 MNIST 作为一个样例.

% ## 推荐随后阅读: <a class="md-anchor" id="AUTOGENERATED-recommended-next-steps-"></a>%

% * [下载与安装](../get_started/os_setup.md)
% * [基本使用](../get_started/basic_usage.md)
% * [TensorFlow 技术指南](../tutorials/mnist/tf/index.md)%

% <div class='sections-order' style="display: none;">
% <!-- os_setup.md -->
% <!-- basic_usage.md -->
% </div>%

% > 原文:[Introduction](http://tensorflow.org/get_started)  翻译:[@doc001](https://github.com/PFZheng)  校对:[@yangtze](https://github.com/sstruct)
