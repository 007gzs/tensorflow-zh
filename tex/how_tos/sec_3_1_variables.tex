%!TEX program = xelatex
% Encoding: UTF8
% SEIKA 2015


% Chapter 3 How to ...
% Section 3.1

\chapter{运作方式}

\section{变量:创建、初始化、保存和加载}

当训练模型时,用变量来存储和更新参数。变量包含张量 (Tensor)存放于内存的缓存区。建模时它们需要被明确地初始化,模型训练后它们必须被存储到磁盘。这些变量的值可在之后模型训练和分析是被加载。

本文档描述以下两个TensorFlow类。点击以下链接可查看完整的API文档:
\begin{itemize}
  \item tf.Variable 类 % add link here
  \item tf.train.Saver 类 % add link here
\end{itemize}

\subsection {创建}

当创建一个变量时,你将一个张量作为初始值传入构造函数Variable()。TensorFlow提供了一系列操作符来初始化张量,初始值是常量或是随机值。
% add link here

\begin{lstlisting}
# Create two variables.
weights = tf.Variable(tf.random_normal([784, 200], stddev=0.35), name="weights")
biases = tf.Variable(tf.zeros([200]), name="biases")
\end{lstlisting}

调用tf.Variable()添加一些操作(Op, operation)到graph:
\begin{itemize}
  \item 一个Variable操作存放变量的值。
  \item 一个初始化op将变量设置为初始值。这事实上是一个tf.assign操作。
  \item 初始值的操作,例如示例中对biases变量的zeros操作也被加入了graph。
\end{itemize}
tf.Variable的返回值是Python的tf.Variable类的一个实例。

\subsection {初始化}

变量的初始化必须在模型的其它操作运行之前先明确地完成。最简单的方法就是添加一个给所有变量初始化的操作,并在使用模型之前首先运行那个操作。

你或者可以从检查点文件中重新获取变量值,详见下文。

使用tf.initialize\_all\_variables()添加一个操作对变量做初始化。记得在完全构建好模型并加载之后再运行那个操作。