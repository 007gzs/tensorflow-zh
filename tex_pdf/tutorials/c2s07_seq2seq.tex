\section{Sequence-to-Sequence Models
}\label{sequence-to-sequence-models}

Recurrent neural networks can learn to model language, as already
discussed in the
\href{tensorflow-zh/SOURCE/tutorials/recurrent/index.md}{RNN Tutorial}
(if you did not read it, please go through it before proceeding with
this one). This raises an interesting question: could we condition the
generated words on some input and generate a meaningful response? For
example, could we train a neural network to translate from English to
French? It turns out that the answer is \emph{yes}.

This tutorial will show you how to build and train such a system
end-to-end. You can start by running this binary.

\begin{verbatim}
bazel run -c opt <...>/models/rnn/translate/translate.py
  --data_dir [your_data_directory]
\end{verbatim}

It will download English-to-French translation data from the
\href{http://www.statmt.org/wmt15/translation-task.html}{WMT'15 Website}
prepare it for training and train. It takes about 20GB of disk space,
and a while to download and prepare (see
\protect\hyperlink{runux5fit}{later} for details), so you can start and
leave it running while reading this tutorial.

This tutorial references the following files from \texttt{models/rnn}.

\begin{longtable}[c]{@{}ll@{}}
\toprule
\begin{minipage}[b]{0.05\columnwidth}\raggedright\strut
File
\strut\end{minipage} &
\begin{minipage}[b]{0.05\columnwidth}\raggedright\strut
What's in it?
\strut\end{minipage}\tabularnewline
\midrule
\endhead
\begin{minipage}[t]{0.05\columnwidth}\raggedright\strut
\texttt{seq2seq.py}
\strut\end{minipage} &
\begin{minipage}[t]{0.05\columnwidth}\raggedright\strut
Library for building sequence-to-sequence models.
\strut\end{minipage}\tabularnewline
\begin{minipage}[t]{0.05\columnwidth}\raggedright\strut
\texttt{translate/seq2seq\_model.py}
\strut\end{minipage} &
\begin{minipage}[t]{0.05\columnwidth}\raggedright\strut
Neural translation sequence-to-sequence model.
\strut\end{minipage}\tabularnewline
\begin{minipage}[t]{0.05\columnwidth}\raggedright\strut
\texttt{translate/data\_utils.py}
\strut\end{minipage} &
\begin{minipage}[t]{0.05\columnwidth}\raggedright\strut
Helper functions for preparing translation data.
\strut\end{minipage}\tabularnewline
\begin{minipage}[t]{0.05\columnwidth}\raggedright\strut
\texttt{translate/translate.py}
\strut\end{minipage} &
\begin{minipage}[t]{0.05\columnwidth}\raggedright\strut
Binary that trains and runs the translation model.
\strut\end{minipage}\tabularnewline
\bottomrule
\end{longtable}

\subsection{Sequence-to-Sequence Basics
}\label{sequence-to-sequence-basics}

A basic sequence-to-sequence model, as introduced in
\href{http://arxiv.org/pdf/1406.1078v3.pdf}{Cho et al., 2014}, consists
of two recurrent neural networks (RNNs): an \emph{encoder} that
processes the input and a \emph{decoder} that generates the output. This
basic architecture is depicted below.

Each box in the picture above represents a cell of the RNN, most
commonly a GRU cell or an LSTM cell (see the
\href{tensorflow-zh/SOURCE/tutorials/recurrent/index.md}{RNN Tutorial}
for an explanation of those). Encoder and decoder can share weights or,
as is more common, use a different set of parameters. Mutli-layer cells
have been successfully used in sequence-to-sequence models too, e.g.~for
translation \href{http://arxiv.org/abs/1409.3215}{Sutskever et al.,
2014}.

In the basic model depicted above, every input has to be encoded into a
fixed-size state vector, as that is the only thing passed to the
decoder. To allow the decoder more direct access to the input, an
\emph{attention} mechanism was introduced in
\href{http://arxiv.org/abs/1409.0473}{Bahdanu et al., 2014}. We will not
go into the details of the attention mechanism (see the paper), suffice
it to say that it allows the decoder to peek into the input at every
decoding step. A multi-layer sequence-to-sequence network with LSTM
cells and attention mechanism in the decoder looks like this.

\subsection{TensorFlow seq2seq Library
}\label{tensorflow-seq2seq-library}

As you can see above, there are many different sequence-to-sequence
models. Each of these models can use different RNN cells, but all of
them accept encoder inputs and decoder inputs. This motivates the
interfaces in the TensorFlow seq2seq library
(\texttt{models/rnn/seq2seq.py}). The basic RNN encoder-decoder
sequence-to-sequence model works as follows.

\begin{Shaded}
\begin{Highlighting}[]
\NormalTok{outputs, states }\OperatorTok{=} \NormalTok{basic_rnn_seq2seq(encoder_inputs, decoder_inputs, cell)}
\end{Highlighting}
\end{Shaded}

In the above call, \texttt{encoder\_inputs} are a list of tensors
representing inputs to the encoder, i.e., corresponding to the letters
\emph{A, B, C} in the first picture above. Similarly,
\texttt{decoder\_inputs} are tensors representing inputs to the decoder,
\emph{GO, W, X, Y, Z} on the first picture.

The \texttt{cell} argument is an instance of the
\texttt{models.rnn.rnn\_cell.RNNCell} class that determines which cell
will be used inside the model. You can use an existing cell, such as
\texttt{GRUCell} or \texttt{LSTMCell}, or you can write your own.
Moreover, \texttt{rnn\_cell} provides wrappers to construct multi-layer
cells, add dropout to cell inputs or outputs, or to do other
transformations, see the
\href{tensorflow-zh/SOURCE/tutorials/recurrent/index.md}{RNN Tutorial}
for examples.

The call to \texttt{basic\_rnn\_seq2seq} returns two arguments:
\texttt{outputs} and \texttt{states}. Both of them are lists of tensors
of the same length as \texttt{decoder\_inputs}. Naturally,
\texttt{outputs} correspond to the outputs of the decoder in each
time-step, in the first picture above that would be \emph{W, X, Y, Z,
EOS}. The returned \texttt{states} represent the internal state of the
decoder at every time-step.

In many applications of sequence-to-sequence models, the output of the
decoder at time t is fed back and becomes the input of the decoder at
time t+1. At test time, when decoding a sequence, this is how the
sequence is constructed. During training, on the other hand, it is
common to provide the correct input to the decoder at every time-step,
even if the decoder made a mistake before. Functions in
\texttt{seq2seq.py} support both modes using the \texttt{feed\_previous}
argument. For example, let's analyze the following use of an embedding
RNN model.

\begin{Shaded}
\begin{Highlighting}[]
\NormalTok{outputs, states }\OperatorTok{=} \NormalTok{embedding_rnn_seq2seq(}
    \NormalTok{encoder_inputs, decoder_inputs, cell,}
    \NormalTok{num_encoder_symbols, num_decoder_symbols,}
    \NormalTok{output_projection}\OperatorTok{=}\VariableTok{None}\NormalTok{, feed_previous}\OperatorTok{=}\VariableTok{False}\NormalTok{)}
\end{Highlighting}
\end{Shaded}

In the \texttt{embedding\_rnn\_seq2seq} model, all inputs (both
\texttt{encoder\_inputs} and \texttt{decoder\_inputs}) are
integer-tensors that represent discrete values. They will be embedded
into a dense representation (see the
\href{tensorflow-zh/SOURCE/tutorials/word2vec/index.md}{Vectors
Representations Tutorial} for more details on embeddings), but to
construct these embeddings we need to specify the maximum number of
discrete symbols that will appear: \texttt{num\_encoder\_symbols} on the
encoder side, and \texttt{num\_decoder\_symbols} on the decoder side.

In the above invocation, we set \texttt{feed\_previous} to False. This
means that the decoder will use \texttt{decoder\_inputs} tensors as
provided. If we set \texttt{feed\_previous} to True, the decoder would
only use the first element of \texttt{decoder\_inputs}. All other
tensors from this list would be ignored, and instead the previous output
of the encoder would be used. This is used for decoding translations in
our translation model, but it can also be used during training, to make
the model more robust to its own mistakes, similar to
\href{http://arxiv.org/pdf/1506.03099v2.pdf}{Bengio et al., 2015}.

One more important argument used above is \texttt{output\_projection}.
If not specified, the outputs of the embedding model will be tensors of
shape batch-size by \texttt{num\_decoder\_symbols} as they represent the
logits for each generated symbol. When training models with large output
vocabularies, i.e., when \texttt{num\_decoder\_symbols} is large, it is
not practical to store these large tensors. Instead, it is better to
return smaller output tensors, which will later be projected onto a
large output tensor using \texttt{output\_projection}. This allows to
use our seq2seq models with a sampled softmax loss, as described in
\href{http://arxiv.org/pdf/1412.2007v2.pdf}{Jean et. al., 2015}.

In addition to \texttt{basic\_rnn\_seq2seq} and
\texttt{embedding\_rnn\_seq2seq} there are a few more
sequence-to-sequence models in \texttt{seq2seq.py}, take a look there.
They all have similar interfaces, so we will not describe them in
detail. We will use \texttt{embedding\_attention\_seq2seq} for our
translation model below.

\subsection{Neural Translation Model }\label{neural-translation-model}

While the core of the sequence-to-sequence model is constructed by the
functions in \texttt{models/rnn/seq2seq.py}, there are still a few
tricks that are worth mentioning that are used in our translation model
in \texttt{models/rnn/translate/seq2seq\_model.py}.

\subsubsection{Sampled softmax and output projection
}\label{sampled-softmax-and-output-projection}

For one, as already mentioned above, we want to use sampled softmax to
handle large output vocabulary. To decode from it, we need to keep track
of the output projection. Both the sampled softmax loss and the output
projections are constructed by the following code in
\texttt{seq2seq\_model.py}.

\begin{Shaded}
\begin{Highlighting}[]
  \ControlFlowTok{if} \NormalTok{num_samples }\OperatorTok{>} \DecValTok{0} \OperatorTok{and} \NormalTok{num_samples }\OperatorTok{<} \VariableTok{self}\NormalTok{.target_vocab_size:}
    \NormalTok{w }\OperatorTok{=} \NormalTok{tf.get_variable(}\StringTok{"proj_w"}\NormalTok{, [size, }\VariableTok{self}\NormalTok{.target_vocab_size])}
    \NormalTok{w_t }\OperatorTok{=} \NormalTok{tf.transpose(w)}
    \NormalTok{b }\OperatorTok{=} \NormalTok{tf.get_variable(}\StringTok{"proj_b"}\NormalTok{, [}\VariableTok{self}\NormalTok{.target_vocab_size])}
    \NormalTok{output_projection }\OperatorTok{=} \NormalTok{(w, b)}

    \KeywordTok{def} \NormalTok{sampled_loss(inputs, labels):}
      \NormalTok{labels }\OperatorTok{=} \NormalTok{tf.reshape(labels, [}\OperatorTok{-}\DecValTok{1}\NormalTok{, }\DecValTok{1}\NormalTok{])}
      \ControlFlowTok{return} \NormalTok{tf.nn.sampled_softmax_loss(w_t, b, inputs, labels, num_samples,}
                                        \VariableTok{self}\NormalTok{.target_vocab_size)}
\end{Highlighting}
\end{Shaded}

First, note that we only construct a sampled softmax if the number of
samples (512 by default) is smaller that the target vocabulary size. For
vocabularies smaller than 512 it might be a better idea to just use a
standard softmax loss.

Then, as you can see, we construct an output projection. It is a pair,
consisting of a weight matrix and a bias vector. If used, the rnn cell
will return vectors of shape batch-size by \texttt{size}, rather than
batch-size by \texttt{target\_vocab\_size}. To recover logits, we need
to multiply by the weight matrix and add the biases, as is done in lines
124-126 in \texttt{seq2seq\_model.py}.

\begin{Shaded}
\begin{Highlighting}[]
\ControlFlowTok{if} \NormalTok{output_projection }\OperatorTok{is} \OperatorTok{not} \VariableTok{None}\NormalTok{:}
  \VariableTok{self}\NormalTok{.outputs[b] }\OperatorTok{=} \NormalTok{[tf.matmul(output, output_projection[}\DecValTok{0}\NormalTok{]) }\OperatorTok{+}
                     \NormalTok{output_projection[}\DecValTok{1}\NormalTok{] }\ControlFlowTok{for} \NormalTok{...]}
\end{Highlighting}
\end{Shaded}

\subsubsection{Bucketing and padding }\label{bucketing-and-padding}

In addition to sampled softmax, our translation model also makes use of
\emph{bucketing}, which is a method to efficiently handle sentences of
different lengths. Let us first clarify the problem. When translating
English to French, we will have English sentences of different lengths
L1 on input, and French sentences of different lengths L2 on output.
Since the English sentence is passed as \texttt{encoder\_inputs}, and
the French sentence comes as \texttt{decoder\_inputs} (prefixed by a GO
symbol), we should in principle create a seq2seq model for every pair
(L1, L2+1) of lengths of an English and French sentence. This would
result in an enormous graph consisting of many very similar subgraphs.
On the other hand, we could just pad every sentence with a special PAD
symbol. Then we'd need only one seq2seq model, for the padded lengths.
But on shorter sentence our model would be inefficient, encoding and
decoding many PAD symbols that are useless.

As a compromise between contructing a graph for every pair of lengths
and padding to a single length, we use a number of \emph{buckets} and
pad each sentence to the length of the bucket above it. In
\texttt{translate.py} we use the following default buckets.

\begin{Shaded}
\begin{Highlighting}[]
\NormalTok{buckets }\OperatorTok{=} \NormalTok{[(}\DecValTok{5}\NormalTok{, }\DecValTok{10}\NormalTok{), (}\DecValTok{10}\NormalTok{, }\DecValTok{15}\NormalTok{), (}\DecValTok{20}\NormalTok{, }\DecValTok{25}\NormalTok{), (}\DecValTok{40}\NormalTok{, }\DecValTok{50}\NormalTok{)]}
\end{Highlighting}
\end{Shaded}

This means that if the input is an English sentence with 3 tokens, and
the corresponding output is a French sentence with 6 tokens, then they
will be put in the first bucket and padded to length 5 for encoder
inputs, and length 10 for decoder inputs. If we have an English sentence
with 8 tokens and the corresponding French sentence has 18 tokens, then
they will not fit into the (10, 15) bucket, and so the (20, 25) bucket
will be used, i.e.~the English sentence will be padded to 20, and the
French one to 25.

Remember that when constructing decoder inputs we prepend the special
\texttt{GO} symbol to the input data. This is done in the
\texttt{get\_batch()} function in \texttt{seq2seq\_model.py}, which also
reverses the input English sentence. Reversing the inputs was shown to
improve results for the neural translation model in
\href{http://arxiv.org/abs/1409.3215}{Sutskever et al., 2014}. To put it
all together, imagine we have the sentence ``I go.'', tokenized as
\texttt{{[}"I",\ "go",\ "."{]}} as input and the sentence ``Je vais.''
as output, tokenized \texttt{{[}"Je",\ "vais",\ "."{]}}. It will be put
in the (5, 10) bucket, with encoder inputs representing
\texttt{{[}PAD\ PAD\ "."\ "go"\ "I"{]}} and decoder inputs
\texttt{{[}GO\ "Je"\ "vais"\ "."\ EOS\ PAD\ PAD\ PAD\ PAD\ PAD{]}}.

\subsection{Let's Run It }\label{lets-run-it}

To train the model described above, we need to a large English-French
corpus. We will use the \emph{10\^{}9-French-English corpus} from the
\href{http://www.statmt.org/wmt15/translation-task.html}{WMT'15 Website}
for training, and the 2013 news test from the same site as development
set. Both data-sets will be downloaded to \texttt{data\_dir} and
training will start, saving checkpoints in \texttt{train\_dir}, when
this command is run.

\begin{verbatim}
bazel run -c opt <...>/models/rnn/translate:translate
  --data_dir [your_data_directory] --train_dir [checkpoints_directory]
  --en_vocab_size=40000 --fr_vocab_size=40000
\end{verbatim}

It takes about 18GB of disk space and several hours to prepare the
training corpus. It is unpacked, vocabulary files are created in
\texttt{data\_dir}, and then the corpus is tokenized and converted to
integer ids. Note the parameters that determine vocabulary sizes. In the
example above, all words outside the 40K most common ones will be
converted to an \texttt{UNK} token representing unknown words. So if you
change vocabulary size, the binary will re-map the corpus to token-ids
again.

After the data is prepared, training starts. Default parameters in
\texttt{translate} are set to quite large values. Large models trained
over a long time give good results, but it might take too long or use
too much memory for your GPU. You can request to train a smaller model
as in the following example.

\begin{verbatim}
bazel run -c opt <...>/models/rnn/translate:translate
  --data_dir [your_data_directory] --train_dir [checkpoints_directory]
  --size=256 --num_layers=2 --steps_per_checkpoint=50
\end{verbatim}

The above command will train a model with 2 layers (the default is 3),
each layer with 256 units (default is 1024), and will save a checkpoint
every 50 steps (the default is 200). You can play with these parameters
to find out how large a model can be to fit into the memory of your GPU.

During training, every \texttt{steps\_per\_checkpoint} steps the binary
will print out statistics from recent steps. With the default parameters
(3 layers of size 1024), first messages look like this.

\begin{verbatim}
global step 200 learning rate 0.5000 step-time 1.39 perplexity 1720.62
  eval: bucket 0 perplexity 184.97
  eval: bucket 1 perplexity 248.81
  eval: bucket 2 perplexity 341.64
  eval: bucket 3 perplexity 469.04
global step 400 learning rate 0.5000 step-time 1.38 perplexity 379.89
  eval: bucket 0 perplexity 151.32
  eval: bucket 1 perplexity 190.36
  eval: bucket 2 perplexity 227.46
  eval: bucket 3 perplexity 238.66
\end{verbatim}

You can see that each step takes just under 1.4 seconds, the perplexity
on the training set and the perplexities on the development set for each
bucket. After about 30K steps, we see perplexities on short sentences
(bucket 0 and 1) going into single digits. Since the training corpus
contains \textasciitilde{}22M sentences, one epoch (going through the
training data once) takes about 340K steps with batch-size of 64. At
this point the model can be used for translating English sentences to
French using the \texttt{-\/-decode} option.

\begin{verbatim}
bazel run -c opt <...>/models/rnn/translate:translate --decode
  --data_dir [your_data_directory] --train_dir [checkpoints_directory]

Reading model parameters from /tmp/translate.ckpt-340000
>  Who is the president of the United States?
 Qui est le président des États-Unis ?
\end{verbatim}

\subsection{What Next? }\label{what-next}

The example above shows how you can build your own English-to-French
translator, end-to-end. Run it and see how the model performs for
yourself. While it has reasonable quality, the default parameters will
not give you the best translation model. Here are a few things you can
improve.

First of all, we use a very promitive tokenizer, the
\texttt{basic\_tokenizer} function in \texttt{data\_utils}. A better
tokenizer can be found on the
\href{http://www.statmt.org/wmt15/translation-task.html}{WMT'15
Website}. Using that tokenizer, and a larger vocabulary, should improve
your translations.

Also, the default parameters of the translation model are not tuned. You
can try changing the learning rate, decay, or initializing the weights
of your model in a different way. You can also change the default
\texttt{GradientDescentOptimizer} in \texttt{seq2seq\_model.py} to a
more advanced one, such as \texttt{AdagradOptimizer}. Try these things
and see how they improve your results!

Finally, the model presented above can be used for any
sequence-to-sequence task, not only for translation. Even if you want to
transform a sequence to a tree, for example to generate a parsing tree,
the same model as above can give state-of-the-art results, as
demonstrated in \href{http://arxiv.org/abs/1412.7449}{Vinyals \& Kaiser
et al., 2015}. So you can not only build your own translator, you can
also build a parser, a chat-bot, or any program that comes to your mind.
Experiment!